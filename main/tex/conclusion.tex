\theoremstyle{plain}
\newtheorem{thmBoolIsCharOfSpec}{Theorem}[section]

\theoremstyle{remark}
\newtheorem*{remarkTheOtherMaximalityConditions}{Remark}
\newtheorem*{remarkShorterProof}{Remark}

%------------------

It is more or less clear that $\charalg{X}$ can be extended
to a functor
\begin{align*}
	\charalg{\cdot} & \,:\,\mathbf{Stone}\,\rightarrow\,
		\mathbf{BoolAlg}
	\text{ .}
\end{align*}
%
But this is not so for the corresponding map $\ufilters{B}$:
for instance, if $f:\,B\rightarrow B'$, although
$f(\phi)=\{f(x)\,:\,x\in\phi\}$ is closed under finite meets
and can, thus, be extended to a filter $\langle f(\phi)\rangle$
in a specified way, it is not clear that this filter is either
maximal or, even, proper. It seems there is no obvious way of
extending $\ufilters{B}$ to a functor in the direction opposite
to that of $\charalg{\cdot}$. At least not if we wanted a
\emph{covariant} functor.

\subsection{An Adjunction}
Clearly the above paragraph made no sense\dots Let $X,Y$ be
compact Hausdorff and totally disconnected spaces. Let
$g:\,X\rightarrow Y$ be a continuous funtion. Then the assignment
$f(U) =g^{-1}(U)$ determines a lattice homomorphism in the
opposite direction: $f:\,\charalg{Y}\rightarrow\charalg{X}$, since
\begin{align*}
	g^{-1}(U\cap V) & \,=\, g^{-1}(U)\cap g^{-1}(V) \text{ ,}\\
	g^{-1}(U\cup V) & \,=\, g^{-1}(U)\cup g^{-1}(V) \text{ ,}\\
	g^{-1}(\varnothing) & \,=\, \varnothing \quad\text{and}\\
	g^{-1}(Y) & \,=\, X
	\text{ .}
\end{align*}
%
In particular this lattice homomorphism is a Boolean algebra
homomorphism. This map $g\mapsto \charalg{g}=f$ respects identities
and compositions. Thus $\charalg{\cdot}$ is a contravariant functor
from the category of Stone spaces to the category of Boolean
algebras.

We now want to extend $\ufilters{B}$ to a functor. As we noted above,
it makes no sense to try to extend $\ufilters{B}$ to a covariant functor.
We shall define a contravariant functor which is given by
$B\mapsto\ufilters{B}$ on objects.

Let $A,B$ be two Boolean algebras and let $f:\,A\rightarrow B$ be a
Boolean algebra homomorphism. If $\phi\subset B$ be a proper subset,
then $f^{-1}(\phi)$ is a proper subset of $A$, too. If $\phi$ is a filter
in $B$, its inverse image $f^{-1}(\phi)$ is a filter in $A$, for
\begin{align*}
	f(x\boolprod y) & \,=\, f(x)\boolprod f(y)\quad\text{and} \\
	y\geq x & \,\Rightarrow\, f(y)\geq f(x)
	\text{ .}
\end{align*}
%
If $\phi$ is a prime filter and $x\boolsum y\in f^{-1}(\phi)$, then
$f(x)\boolsum f(y)$ belongs to $\phi$ and, thus, either $f(x)\in\phi$ or
$f(y)\in\phi$. So, $f^{-1}(\phi)$ is prime, as well. Since $A$ and $B$ are
Boolean algebras, the notions of prime filter and of ultrafilter --that is,
maximal filter-- agree.

\begin{remarkTheOtherMaximalityConditions}
	We may wonder if the other equivalent maximality conditions
	for a filter in a Boolean algebra are preserved under the
	operation of taking inverse images.

	If $\phi$ is a proper filter such that, for every $z\in B$
	either $z\in\phi$ or $\conj{z}\in\phi$, then, given $x\in A$,
	we have $f(x)\in\phi$, if and only if $f(\conj{x})\not\in\phi$.
	Thus, $f^{-1}(\phi)$ has the property that, for every $x\in A$,
	either $x\in f^{-1}(\phi)$ or $\conj{x}\in f^{-1}(\phi)$, but not
	both. Thus, we see that the two equivalent conditons to being a
	maximal filter are preserved under the operation of taking inverse
	image by a lattice homomorphims.

	If $A$ and $B$ were arbitrary lattices, it would not be true that
	the inverse image of a maximal filter is a maximal filter. However,
	if $f$ were surjective, the ``forward'' image of a filter would be
	a filter and it would be true that the nverse image of a maximal
	filter is a maximal filter.
\end{remarkTheOtherMaximalityConditions}

To summarise, if $f:\,A\rightarrow B$ is a Boolean algebra homomorphism and
$\phi$ is an ultrafilter in $B$, the function $g(\phi)=f^{-1}(\phi)\subset A$
determines a map $g:\,\ufilters{B}\rightarrow\ufilters{A}$. We check now that
this map is continuous.

A base for the topology on $\ufilters{A}$ is given by the sets of the form
$u_{A}(x)=\{\psi\in\ufilters{A}\,:\,x\in\psi\}$ with $x\in A$. Let $x\in A$
be an arbitrary element. Then
\begin{align*}
	g^{-1}(u_{A}(x)) & \,=\,\left\lbrace\phi\in\ufilters{B}
		\,:\,x\in f^{-1}(\phi)\right\rbrace \\
	& \,=\,\left\lbrace\phi\in\ufilters{B}
		\,:\,f(x)\in\phi\right\rbrace
		\,=\,u_{B}(f(x))
	\text{ .}
\end{align*}
%
Thus, the preimage of an element of the base $\charalg{\ufilters{A}}$
for the topology on the Stone space of $A$ is an element of the base
$\charalg{\ufilters{B}}$ for the topology on the Stone space of $B$.
In particular, $g:\ufilters{B}\rightarrow\ufilters{A}$ is continuous. The
map $f\mapsto\ufilters{f}=g$ respects identities and compositions, so it
extends to a contravariant functor from the category of Boolean algebras to
the category of Stone spaces.

We thus have two contravariant functors
\begin{center}
	\begin{tikzcd}
		\mathbf{Stone}
			\arrow[r, leftarrow, shift left, "\ufilters{\cdot}"]
			\arrow[r, rightarrow, shift right, "\charalg{\cdot}"']
			&
		\mathbf{BoolAlg}
	\end{tikzcd}
\end{center}
%
Theorem \ref{thm:stoneofboolisstone} shows that the composition
$B\mapsto\charalg{\ufilters{B}}$ is naturally isomorphic to the identity
functor on $\mathbf{BoolAlg}$. This isomorphism is given by the map
$u(x)=\{\phi\in\ufilters{B}\,:\,x\in\phi\}$ for each Boolean algebra $B$.
Similarly, theorem \ref{thm:stoneisstoneofbool} shows that
$X\mapsto\ufilters{\charalg{X}}$ is naturally isomorphic to the identity
functor in $\mathbf{Stone}$. The isomorphism in this case is given by the
corresponding function $v(\xi)=\{U\in\charalg{X}\,:\,\xi\in U\}$.

\subsection{The Relation with the Prime Spectrum}
Let $B$ be a Boolean algebra. We have already seen that the lattice
ideals are in corresppondence with the ring-theoretic ideals of $B$ seen
as a Boolean ring. In fact, the correspondence is given by the identity
map as subsets of $B$. The same is true for prime ideals. Therefore, we may
conclude that ring-theoretic prime ideals in $B$ are in correspondence with
lattice ideals in $B$ (by the identity map
$\id:\,\powset{B}\rightarrow\powset{B}$ on subsets), which are in
correspondence with prime filters in $B$ (by taking set-theoretic complements
$I\mapsto\setcomp{I}=B\setmin I$). In particular, ring-theoretic prime ideals
are exactly the ring-theoretic maximal ideals in $B$.
All ring-theoretic prime ideals of a Boolean ring $B$ are kernels of
ring homomorphisms $f:\,B\rightarrow\thetwo$: the canonical projection
$B\rightarrow B/I$ is a surjective ring homomorphism onto an integral
domain in which every element is idempotent, hence $B/I\simeq\thetwo$.
Every lattice homomorphism between Boolean algebras is a ring homomorphism
between the corresponding Boolean ring structures.

Conversely, every kernel of a ring or lattice homomorphism
$f:\,B\rightarrow\thetwo$ is a prime ideal, since its complement $f^{-1}(1)$
is a filter in $B$. Ring (or lattice) prime ideals in a Boolean algebra $B$
are, therefore, precisely the kernels $f^{-1}(0)$ of ring (or lattice)
homomorphisms $f:\,B\rightarrow\thetwo$ and ultrafilters (equivalently,
prime filters) are, precisely, the \emph{shells} $f^{-1}(1)$ of such
homomorphisms. Thus, the space of ultrafilters in a Boolean algebra is
in correspondence with the \emph{prime spectrum} of the Boolean ring $B$:
\begin{center}
	\begin{tikzcd}
		\ufilters{B}\arrow[r,"\setcomp{\cdot}"] &
			\begin{Bmatrix}
				\text{prime ideals} \\
				\text{in $B$}
			\end{Bmatrix} \arrow[r,"\id"] &
			\spec{B} \\
		\phi \arrow[r,maps to] &
			I=\setcomp{\phi}=B\setmin\phi \arrow[r,maps to] &
			I
	\end{tikzcd}
\end{center}
%
Consider the extremes of the above diagram. On the one hand, we have
given $\ufilters{B}$ the topology determined by the base $u(B)$, that is,
a basic open (closed) set is of the form
\begin{align*}
	u(x) & \,=\,\left\lbrace\phi\in\ufilters{B}\,:\,x\in\phi\right\rbrace
	\text{ ,}
\end{align*}
%
and we have proved that, with this topology, $\ufilters{B}$ is a Stone
space. We have also proved that $B$ can be ``recovered'' as the
characteristic algebra of closed and open subsets of $\ufilters{B}$.
On the other hand, the topology on $\spec{B}$ is determined by the base
formed by the sets
\begin{align*}
	D_{x} & \,=\,\left\lbrace I\subset B\,:\,
		I\text{ is a prime ideal},\,x\not\in I\right\rbrace
		\,=\,\spec{B}\setmin V(x)
	\text{ ,}
\end{align*}
%
where $V(x)$ is the set of (ring-theoretic) prime ideals of $B$ containing
$x$. By the bijection between ring prime ideals, lattice prime ideals and
ultrafilters in a Boolean algebra, the sets $D_{x}$ are equal to
\begin{equation}
	\label{eq:zariskibasicopen}
	\begin{aligned}
		D_{x} & \,=\,\left\lbrace\setcomp{\phi}\,:\,
				\phi\in\ufilters{B},\,x\in\phi\right\rbrace \\
		& \,=\,\left\lbrace I\in\spec{X}\,:\,
				\setcomp{I}\in u(x)\right\rbrace
	\text{ .}
	\end{aligned}
\end{equation}
%
In general, for any commutative ring with unit $B$, if $x,y\in B$,
\begin{align*}
	D_{x}\cap D_{y} & \,=\, D_{x\cdot y}\text{ ,} \\
	D_{0} & \,=\, B\quad\text{and} \\
	D_{1} & \,=\, \varnothing
	\text{ .}
\end{align*}
%
So, $\{D_{x}\,:x\in B\}$ forms a base for a topology on $\spec{B}$.
Furthermore, in a Boolean ring, we see, by \eqref{eq:zariskibasicopen}, that
\begin{align*}
	\spec{B}\setmin D_{x} & \,=\, D_{\conj{x}}\quad\text{and} \\
	D_{x}\cup D_{y} & \,=\,
		(\spec{B}\setmin D_{\conj{x}})\cup
			(\spec{B}\setmin D_{\conj{y}})
		\,=\,D_{\conj{\lconj{\conj{x}\cdot\conj{y}}}}
	\text{ .}
\end{align*}
%
Thus, the collection of sets $D_{x}$ are both open and closed in the topology
they generate. In fact, these are the only open and closed subsets of
$\spec{B}$: if $V\subset\spec{B}$ is an open and closed subset, being open,
there exists a collection of basic open sets $\{D_{x_{i}}\,:\,i\in I\}$
such that their union is $V$:
\begin{align*}
	V & \,=\,\bigcup\,\left\lbrace D_{x_{i}}\,:\,i\in I\right\rbrace \\
	& \,=\,	\Bigg\lbrace I\in\spec{B} \,:\,\setcomp{I}\in
			\bigcup_{i\in I}\,u(x_{i})\Bigg\rbrace
	\text{ ,}
\end{align*}
%
and, being closed, ther exists a collection $\{D_{x_{j}}\,:\,j\in J\}$
such that their intersection is $V$:
\begin{align*}
	V & \,=\,\bigcap\,\left\lbrace D_{x_{i}}\,:\,i\in I\right\rbrace \\
	& \,=\,	\Bigg\lbrace I\in\spec{B} \,:\,\setcomp{I}\in
			\bigcap_{j\in J}\,u(x_{j})\Bigg\rbrace
	\text{ .}
\end{align*}
%
Since $I\mapsto\setcomp{I}$ is bijective, the set
\begin{align*}
	U & \,:=\, \bigcup_{i\in I}\,u(x_{i}) \,=\,\bigcap_{j\in J}\,u(x_{j})
\end{align*}
%
is both open and closed in $\ufilters{B}$. But since $\ufilters{B}$ is
compact and $U$ is closed, $U$ must be compact. Therefore,
\begin{align*}
	U & \,=\,u(x_{i_{1}})\cup\,\cdots\,\cup u(x_{i_{k}})
\end{align*}
%
and
\begin{align*}
	V & \,=\,D_{x_{i_{1}}}\cup\,\cdots\,\cup D_{x_{i_{k}}}
\end{align*}
%
But then $V$ is equal to $D_{y}$ for some $y\in B$.

\begin{remarkShorterProof}
	Another --perhaps shorter-- proof of this fact consists in showing
	first that $\spec{B}$ is, in general, compact (though maybe not
	Hausdorff). If $V\subset\spec{B}$ is both open and closed, being
	open it is a union of basic open subsets $D_{x}$ and, being closed,
	it is compact and the cover can be replaced by a finite subcover
	$\{D_{x_{1}},\,\dots,\,D_{x_{k}}\}$. Since a finite union of
	these basic open subsets is, in a Boolean ring, equal to some
	basic open subset $D_{y}$, we conclude that $V= D_{y}$ and that
	the basic open subsets $D_{x}$ are the only open and closed subsets
	of $\spec{B}$. This proof is rather similar to the proof of
	\ref{thm:charalg}.
\end{remarkShorterProof}

We can then prove the following theorem, analogous to
\ref{thm:stoneofboolisstone}.

\begin{thmBoolIsCharOfSpec}\label{thm:boolischarofspec}
	Let $B$ be a Boolean ring. The prime spectrum of $B$, $\spec{B}$,
	with its Zariski topology determined by the sets $D_{x}$ $x\in B$
	is a Stone space, a compact Hausdorff totally disconnected
	topological space. The map $D:\,B\rightarrow\powset{\spec{B}}$
	given by
	\begin{align*}
		D(x) & \,=\, D_{x} \,=\,\left\lbrace I\in\spec{B}\,:\,
			x\not\in I\right\rbrace
	\end{align*}
	%
	determines a Boolean algebra isomorphism from $B$ onto the
	characteristic algebra $\charalg{\spec{B}}$ of open and closed
	subsets of the prime spectrum of $B$.
\end{thmBoolIsCharOfSpec}

