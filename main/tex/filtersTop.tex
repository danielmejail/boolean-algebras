\theoremstyle{plain}
\newtheorem{thmBoolPowsetIso}{Proposition}[section]
\newtheorem{lemmaFilterConverges}[thmBoolPowsetIso]{Lemma}
\newtheorem{coroTopoConseq}[thmBoolPowsetIso]{Corollary}
\newtheorem{lemmaCharAlg}[thmBoolPowsetIso]{Lemma}
\newtheorem{thmStoneSpaceOfBoolIsStoneSpace}[thmBoolPowsetIso]{Theorem}
\newtheorem{thmStoneSpaceIsStoneSpaceOfBool}[thmBoolPowsetIso]{Theorem}

\theoremstyle{remark}

%------------------


We begin this section defining the neighbourhood filter.

Let $X$ be a topologial space nad let $x\in X$ be any point. The
\emph{neighbourhood filter} at $x$ \emph{on} $X$ is the filter
$N_{x}$ in $\powset{X}$ (partially ordered by inclusion) consisting of
precisely the open subsets of $X$ containing $x$.

\begin{lemmaFilterConverges}\label{thm:filterconverges}
	Given a point $x$ in a topological space $X$, the following conditions
	on a filter $F$ on $X$ are equivalent:
	\begin{itemize}
		\item[i]
			\begin{math}
				x\in\bigcup\left\lbrace\clos{A}
				\,:\,A\in F\right\rbrace
			\end{math};
		\item[ii] if $A\in F$ and $U\in N_{x}$, then
			$A\cap U\not=\varnothing$; and
		\item[iii] the union $F\cup N_{x}$ is contained in a (proper)
			filter in $\powset{X}$.
	\end{itemize}
	%
\end{lemmaFilterConverges}

\begin{proof}
	The first two are equivalent by definition of closure of a set.
	\textit{(ii)} is equivalent to $F\cup N_{x}$ having the finite
	intersection property, which, by \ref{thm:filtergeneratedbysubset},
	is equivalent to \textit{(iii)}.
\end{proof}

If any of the equivalent conditions in lemma \ref{thm:filterconverges}
is satisfied, we say that $x$ is an \emph{adherent point} of the filter
$F$. If, moreover, $F\supset N_{x}$, then $F$ is said to \emph{converge}
to $x$.

If $F$ converges to a point $x\in X$, then $x$ is an adherent
point of $F$. If $F$ is an ultrafilter and $x$ is an adherent point of
$F$, then $N_{x}$ must be contained in $F$, which implies that $F$
converges to $x$. In other words, an ultrafilter converges to a point, if
and only if this point is an adherent point of the ultrafilter.

\begin{coroTopoConseq}\label{thm:topoconseq}
	Let $X$ be a topological space. Then
	\begin{itemize}
		\item[(a)] $X$ is Hausdorff, if and only
			if every filter converges to at most one point.
			Equivalently, $X$ is Hausdorff, if and only if every
			ultrafilter has at most one adherent point
			(converges to at most one point);
		\item[(b)] $X$ is compact, if and only if each filter has,
			at least, one adherent point. Equivalently, $X$
			is compact, if and only if every ultrafilter has
			at least one adherent point (converges to at least
			one point).
	\end{itemize}
	%
\end{coroTopoConseq}

\subsection{Stone Spaces}
Given a countable set $X$, the Boolean subalgebra $\mathsf{Z}(X)$ of finite
and cofinite subsets of $X$ (of the power set Boolean algebra $\powset{X}$
partially ordered by inclusion) is \emph{not} isomorphic to any power set
Boolean algebra, since it is countable.

We see from \ref{thm:topoconseq} that, given a compact Hausdorff topological
space $X$, there is an associated map:
\begin{center}
	\begin{tikzcd}
		\begin{Bmatrix}
			\text{ultrafilters} \\
			\text{on $X$}
		\end{Bmatrix}
		\arrow[r] &
		\begin{Bmatrix}
			\text{points} \\
			\text{in $X$}
		\end{Bmatrix}
	\end{tikzcd}
\end{center}
So we may wonder what kind of structure does the set of ultrafilters on $X$
possess. In oreder to understand this, we consider the case of an arbitrary
Boolean algebra.

Let $B$ be a Boolean algebra and let $\ufilters{B}$ denote the set of
ultrafilters in $B$. Consider the following map $u$ which to each element
$x\in B$ asigns the subset of $\ufilters{B}$ of ultrafilters containing $x$,
that is,
\begin{equation}
	\label{eq:boolpowsetiso}
	\begin{aligned}
		u(x) & \,=\,\left\lbrace\phi\in\ufilters{B}
			\,:\,x\in\phi\right\rbrace
		\text{ .}
	\end{aligned}
\end{equation}
%
Given $x\,y\in B$, if $x\not = y$, then, by \ref{thm:filterseparation},
there exists an ultrafilter $\phi$ containing one but not the other. In
particular, the map $u:\,B\rightarrow\powset{\ufilters{B}}$ is injective.

\begin{thmBoolPowsetIso}\label{thm:boolpowsetiso}
	Let $B$ be a Boolean algebra, $\ufilters{B}$ be the set of filters
	in $B$ and let $\powset{\ufilters{B}}$ be the power set Boolean
	algebra of subsets of $\ufilters{B}$ ordered by inclusion. Let
	$u:\,B\rightarrow\powset{\ufilters{B}}$ be the map defined by
	\eqref{eq:boolpowsetiso}. Then $u$ is an injective Boolean algebra
	homomorphism. In particular, $B$ is isomorphic to a subalgebra of
	$\powset{\ufilters{B}}$.
\end{thmBoolPowsetIso}

\begin{proof}
	If $F$ is any filter and $x,y\in B$, then $x,y\in F$, if and only
	if $x\boolprod y\in F$. This shows that
	\begin{align*}
		u(x\boolprod y) & \,=\,u(x)\cap u(y)
		\text{ ,}
	\end{align*}
	%
	which is equal to the meet of the elements $u(x)$ and $u(y)$
	in $\powset{\ufilters{B}}$. So, $u$ respects meets. By one of
	the equivalent conditions of being an ultrafilter, if
	$\phi\in\ufilters{B}$, for every $x\in B$, $x\in\phi$ xor
	$\conj{x}\in\phi$. Thus,
	\begin{align*}
		u(\conj{x}) & \,=\,\ufilters{B}\setmin u(x)
			\,=\,\setcomp{u(x)}
		\text{ ,}
	\end{align*}
	%
	which is the complement of $u(x)$ in $\powset{\ufilters{B}}$.
	Since no proper filter contains the least element $0$ in $B$,
	$u(0)$ equals the empty subset and, since every filter contains the
	greatest element $1$, $u(1)$ equals $\ufilters{B}$. But these are
	precisely the least and greatest elements in the lattice
	$\powset{\ufilters{B}}$. Then $u$ respects $\boolsum$, as
	well, and is, therefore, a Boolean algebra homomorphism.
\end{proof}

A \emph{Stone space} is a topological space $X$ which is compact, Hausdorff
and totally disconected (equivalently, admits a basis of closed and open
subsets). Given an arbitrary topological space $X$, the collection of all
subsets of $X$ which are both closed and open forms a Boolean algebra, a
subalgebra of the Boolean algebra $\powset{X}$ of subsets of $X$. We shall
call this algebra, the \emph{characteristic algebra} of $X$ and denote it
by $\charalg{X}$.

In a set $X$, if $\cal{A}$ is a subalgebra of $\powset{X}$, then $\cal{A}$
constitutes a base for a topology on $X$. With this topology, the elements of
$\cal{A}$ are both open and closed subsets of $X$. In particular, if $X$ is a
Stone space, then $\charalg{X}$ forms a basis for the topology on $X$. The
following lemma says that the characteristic algebra of a Stone space is
the only Boolean subalgebra of subsets of $X$ with this property.

\begin{lemmaCharAlg}\label{thm:charalg}
	Let $X$ be a Stone space and let $\cal{A}$ be a subalgebra of the
	Boolean algebra $\powset{X}$. If $\cal{A}$ is a basis for the topology
	on $X$, then $\cal{A}=\charalg{X}$.
\end{lemmaCharAlg}

\begin{proof}
	The elements of $\cal{A}$ are both open and closed, since
	$\cal{A}$ is, by hypothesis, a part of the topology on $X$ and
	$\cal{A}$ is, being a subalgebra, closed under complementation.
	Thus, $\cal{A}\subset\charalg{X}$.

	Now let $V\in\charalg{X}$. On the one hand, since $V$ is open and
	$\cal{A}$ is a base for the topology on $X$, given $x\in V$, there
	exists $U_{x}\in\cal{A}$ such that $x\in U_{x}\subset V$. On the
	other hand, since $V$ is closed and $X$ is compact, $V$ is,
	therefore compact. Combining these two observations, we conclude
	that $V$ is covered by finitely many $U_{x}$, so
	\begin{align*}
		V & \,=\, U_{x_{1}}\cup\,\cdots\,\cup U_{x_{k}}
		\text{ .}
	\end{align*}
	%
	This shows that $V$ belongs to $\cal{A}$.
\end{proof}

Given a Boolean algebra $B$, we have seen that the homomorphism
$u:\,B\rightarrow\powset{\ufilters{B}}$ determines an isomorphism between
$B$ and the subalgebra $u(B)$ of $\powset{\ufilters{B}}$. Now, the subalgebra
$u(B)$ determines a topology on $\ufilters{B}$. The topological space
consisting of the set $\ufilters{B}$ together with the topology determined
by the Boolean algebra $B\simeq u(B)\subset\powset{\ufilters{B}}$ is called
the \emph{Stone space of the Boolean algebra $B$}. The following theorem
shows, in particular, that the Stone space of a Boolean algebra is a Stone
space, a compact, Hausdorff and totally disconnected topological space.

\begin{thmStoneSpaceOfBoolIsStoneSpace}\label{thm:stoneofboolisstone}
	The Stone space $\ufilters{B}$ of a Boolean algebra $B$ is a
	Stone space and the map $u:\,B\rightarrow\powset{\ufilters{B}}$
	is an isomorphism from $B$ onto the characteristic algebra
	$\charalg{\ufilters{B}}$ of $\ufilters{B}$.
\end{thmStoneSpaceOfBoolIsStoneSpace}

\begin{proof}
	We show firstly that $\ufilters{B}$ is Hausdorff when given the
	topology determined by $u(B)$. Let $\phi,\psi\in\ufilters{B}$
	be two ultrafilters in $B$ and suppose $\phi\not=\psi$. This
	means $\phi$ and $\psi$ are different as subsets of $B$, so
	there exists $x\in\phi$ such that $x\not\in\psi$ (note that, by
	maximality, both $\psi\setmin\phi$ and $\phi\setmin\psi$ are
	nonempty). In particular, $\phi\in u(x)$ and $\psi\not\in u(x)$.
	By maximality of $\psi$, it follows that $\psi\in u(\conj{x})$.
	Since $u(x)$ and $u(\conj{x})$ are disjoint open subsets, we
	conclude that $\ufilters{B}$ is Hausdorff. Note that we have also
	shown that the space of ultrafilters is totally disconnected.

	To show that $\ufilters{B}$ is compact, consider an open cover
	$\{u(x_{i})\,:\,i\in I\}$ by elements of the base $u(B)$.
	Assume this cover does not admit a finite subcover.
	This is equivalent to assuming the collection of complements
	$\{u(\conj{x_{i}})\,:\,i\in I\}$ has the finite intersection
	property. But then the subset $\{\conj{x_{i}}\,:\,i\in I\}$ has
	the finite intersection property. By
	\ref{thm:filtergeneratedbysubset}, there exists an ultrafilter
	$\phi$ such $\conj{x_{i}}\in\phi$ for every $i\in I$. This implies
	\begin{align*}
		\phi & \,\in\,\bigcap_{i\in I}\,u(\conj{x_{i}})
		\text{ .}
	\end{align*}
	%
	Equivalently, $x_{i}\not\in\phi$ for every $i\in I$ and
	\begin{align*}
		\phi & \,\not\in\,\bigcup_{i\in I}\,u(x_{i})
		\text{ ,}
	\end{align*}
	%
	contradicting the assumption that $\{u(x_{i})\,:\,i\in I\}$ was
	a cover of $\ufilters{B}$.

	The last assertion follows from \ref{thm:charalg}, since $u(B)$ has
	just been shown to be a base for the topology of the Stone space
	$\ufilters{B}$.
\end{proof}

The preivious theorem also showed that every Boolean algebra is the
characteristic algebra of a Stone space, especifically, of its Stone
space. In the opposite direction, every Stone space is homeomorphic to the
Stone space of its characteristic algebra (every characteristic algebra is
the characteristic algebra of the Stone space of a Boolean algebra\dots).

\begin{thmStoneSpaceIsStoneSpaceOfBool}\label{thm:stoneisstoneofbool}
	Let $X$ be a Stone space. Then $X$ is homeomorphic to the
	Stone space $\ufilters{\charalg{X}}$ of its characteristic algebra.
\end{thmStoneSpaceIsStoneSpaceOfBool}

\begin{proof}
	Let $X$ be a Stone space. Assume $X$ is not the one-point space
	$\theone$; the proof is trivial, otherwise. So, assume that
	$X\not=\theone$ and define a map from $X$ to the Stone space of
	its characteristic algebra $v:\,X\rightarrow\ufilters{\charalg{X}}$
	by
	\begin{align*}
		v(\xi) & \,=\, \left\lbrace U\in\charalg{X}
			\,:\, \xi\in U\right\rbrace
		\text{ .}
	\end{align*}
	%
	For each fixed point $\xi$ of $X$, the collection $v(\xi)$
	constitutes a filter in $\charalg{X}$. Since $X$ has more than
	one point and is a Hausdorff space, $v(\xi)$ must be a proper
	subset of $\charalg{X}$. For each open and closed subsets
	$U\in\charalg{X}$ and $\xi\in X$, either $\xi\in U$ or
	$\xi\not\in U$. Equivalently, either $\xi\in U$ or
	$\xi\in X\setmin U=\setcomp{U}$. Thus, for each fixed $x\in X$,
	$v(x)$ is an ultrafilter. Thus $v$ determines a well-defined map
	\begin{align*}
		v & \,:\, X\,\rightarrow\,\ufilters{\charalg{X}}
		\text{ .}
	\end{align*}
	%

	If $\xi,\upsilon\in X$ are distinct elements of $X$, then there
	exist $U,V\in\charalg{X}$ such that $U\cap V=\varnothing$,
	$\xi\in U$ and $\upsilon\in V$. In particular,
	$v(\xi)\not =v(\upsilon)$ (in fact, it is enough to assume $X$ is
	a $T_{0}$ space to conclude that $v$ is injective).

	Let $\phi\in\ufilters{\charalg{X}}$ be an arbitrary ultrafilter in
	$\charalg{X}$. Then $\phi$ has the finite intersection property.
	Since $\charalg{X}$ is a sublattice of the Boolean algebra
	$\powset{X}$ of subsets of $X$, $\phi$ has the finite intersection
	property as a subset of $\powset{X}$ (note the parallelism with
	invariance of compactness under subspace maps). Thus $\phi$
	extends to an ultrafilter $F$ in $\powset{X}$. Since $X$ is a
	compact topological space, $F$ converge to some point $\xi\in X$.
	Equivalently, $\xi$ is an adherent point of $F$. Then
	$\xi\in\bigcap_{U\in F}\,\clos{U}$. In particular,
	\begin{align*}
		\xi & \,\in\,\bigcap\,\left\lbrace \clos{U}
			\,:\, U\in\phi\right\rbrace
		\,=\,\bigcap\,\left\lbrace U
			\,:\, U\in\phi\right\rbrace
		\text{ .}
	\end{align*}
	%
	Thus $\xi\in U$ for ever $U$ in $\phi$ and then $\phi\subset v(\xi)$.
	By maximality of $\phi$ and properness of $v(\xi)$, we may conclude
	that $\phi=v(\xi)$. So, we see $v$ is also surjective.

	Finally, $v:\,X\rightarrow\ufilters{\charalg{X}}$ is a bijective
 	map between compact Hausdorff spaces which maps the base
	$\charalg{X}$ for the topology on $X$ into the base
	$\charalg{\ufilters{\charalg{X}}}$ for the topology on
	$\ufilters{\charalg{X}}$: if $V\in\charalg{X}$, by surjectivity
	of $v$,
	\begin{align*}
		v(V) & \,=\,\left\lbrace
			\left\lbrace U\in\charalg{X}\,:\,\xi\in U
				\right\rbrace
				\,:\,\xi\in V\right\rbrace
			\,=\,\left\lbrace v(\xi)\,:\,\xi\in V
				\right\rbrace \\
		& \,=\, \left\lbrace\phi\in\ufilters{\charalg{X}}
			\,:\,V\in\phi\right\rbrace
		\text{ .}
	\end{align*}
	%
	In particular, $v(V)\in\charalg{\ufilters{\charalg{X}}}$ and
	$v(V)$ is an element of the base for the Stone space of $\charalg{X}$.
	In fact, this shows that $v$ maps the base $\charalg{X}$ of $X$
	onto the base $\charalg{\ufilters{\charalg{X}}}$. It then follows
	that $v$ is a homeomorphism.
\end{proof}

