\theoremstyle{plain}
\newtheorem{thmBoolAlgIsoBoolRng}{Theorem}[section]
\newtheorem{propoAbsorptiveLaws}[thmBoolAlgIsoBoolRng]{Proposition}
\newtheorem{thmJoinSemilat}[thmBoolAlgIsoBoolRng]{Theorem}
\newtheorem{thmJoinSemilatChar}[thmBoolAlgIsoBoolRng]{Theorem}
\newtheorem{thmMeetSemilat}[thmBoolAlgIsoBoolRng]{Theorem}
\newtheorem{thmMeetSemilatChar}[thmBoolAlgIsoBoolRng]{Theorem}
\newtheorem{lemmaDualDistLaw}[thmBoolAlgIsoBoolRng]{Lemma}
\newtheorem{propoDistLat}[thmBoolAlgIsoBoolRng]{Proposition}
\newtheorem{thmBoolAlgIsBoolRng}[thmBoolAlgIsoBoolRng]{Theorem}
\newtheorem{lemmaBoolRngIsDistLat}[thmBoolAlgIsoBoolRng]{Lemma}
\newtheorem{thmBoolRngIsBoolAlg}[thmBoolAlgIsoBoolRng]{Theorem}
\newtheorem{lemmaSymmDiff}[thmBoolAlgIsoBoolRng]{Lemma}

% \newtheorem{propoLatticeJoinsAndMeets}{Proposition}

\theoremstyle{remark}
\newtheorem*{remarkJoinEmptySet}{Remark}
\newtheorem*{remarkMeetEmptySet}{Remark}
\newtheorem*{remarkBoolAlgIsFullInLat}{Remark}


%-----------------------

A \emph{partially ordered set} (also \emph{poset}) is defined to be a set
together with a binary relation, called the \emph{order relation}, which is
\emph{reflexive}, \emph{transitive} and \emph{antisymmetric}. A poset
is said to be \emph{totally ordered}, if the order relation is
\emph{dychotomic}.

Let $(P,\leq)$ be a poset and let $Q\subset P$ be an arbitrary subset. If
we restrict the ordered relation $\leq$ to elements belonging to $Q$, then
a partial order is obtained on $Q$. The pair $(Q,\leq)$ is then a poset, as
well. We shall say in this case that $Q$ is a \emph{subposet} of $P$ or that
$Q$ \emph{inherits} the poset structure --or the order relation-- from $P$.

A \emph{chain} in a partially ordered set is a subset which is totally
ordered by the order relation inherited from the poset.

\paragraph{Examples}
So, we see that $\bb{F}_{2}$, $\powset{X}$ and $\mathsf{Z}(X)$ are
partially ordered sets, if endowed with the order relations defined above.
The two-element set $\bb{F}_{2}$ is totally ordered but, in general,
$\powset{X}$ and $\mathsf{Z}(X)$ are not. 

\subsection{Joins and Meets}
Given a partially ordered set $(P,\leq)$ and a subset $A\subset P$, an
\emph{upper bound} for $A$ in $P$ is an element $x\in P$ such that
\begin{align*}
	a & \,\leq\, x
\end{align*}
%
for all $a\in A$. We shall say that an element $x\in P$ is a \emph{supremum}
(or join) for $A$ in $P$, if it is a least upper bound for $A$ in $P$, that
is to say
\begin{itemize}
	\item[i] $a\leq x$ for all $a\in A$ and
	\item[ii] if $b$ satisfies $a\leq b$ for all $a\in A$, then
		$x\leq b$.
\end{itemize}
%
We denote it by $\bigboolsum\,A$. Upper bounds and therefore joins need not
exist for arbitrary subsets of a poset $P$. By the antisymmetry of $\leq$
in $P$, if a join for a subset $A$ exists, then it must be unique.
Given a poset $(P,\leq)$ and a pair of elements $x,y\in P$, the
\emph{join} of $x$ and $y$ is the supremum (join) for the set $\{x,y\}$ in
$P$, if it exists. We denote it by $\bigboolsum\,\{x,y\}=x\boolsum y$.

\begin{remarkJoinEmptySet}
	Let $(P,\leq)$ be a poset. Every element $x\in P$ is, trivially,
	an upper bound for the empty subset $\varnothing\subset P$. This
	does not imply, however, that there exists a join for $\varnothing$.
	If the empty subset admits a join, $x=\bigboolsum\varnothing$, then
	this means that $x\leq b$ for every element $b\in P$.
\end{remarkJoinEmptySet}

Dually, given a poset $(P,\leq)$ and a subset $A\subset P$, we say that an
element $x\in P$ is a \emph{lower bound} for $A$ in $P$, if
\begin{align*}
	a & \,\geq\, x
\end{align*}
%
for all $a \in A$. An \emph{infimum} (or meet) for $A$ is a lower bound
for $A$ which greatest among the lower bounds for $A$, this means
\begin{itemize}
	\item[i] $a\geq x$ for all $a\in A$ and
	\item[ii] if $b\in P$ satisfies $a\geq b$ for all $a\in A$, then
		$x\geq b$.
\end{itemize}
%
If such an element exists, we shall denote it by $\bigboolprod A$.
Lower bounds, as well as meets, need not exist. However, if a meet for a
subset $A$ exists, then it must be unique by the antisymmetry property
of the order relation $\leq$ in $P$. Given two elements $x,y\in P$,
the \emph{meet} of $x$ and $y$ is, if it exists, the infimum (meet) for
the set $\{x,y\}$ in $P$ and is denoted by
$\bigboolprod\,\{x,y\}=x\boolprod y$.

\begin{remarkMeetEmptySet}
	Let $(P,\leq)$ be a poset. Every element $x\in P$ is, trivially,
	an lower bound for the empty subset $\varnothing\subset P$. This
	does not imply, however, that there exists a meet for $\varnothing$.
	If the empty subset admits a meet, $x=\bigboolprod\varnothing$, then
	this means that $x\geq b$ for every element $b\in P$.
\end{remarkMeetEmptySet}

\paragraph{Example: $\powset{X}$}
In $\powset{X}$, every subset $\cal{A}\subset\powset{X}$ admits a supremum
(as well as an infimum) in $\powset{X}$. In particular, chains in
$\powset{X}$ have upper and lower bounds.

\paragraph{Example: $\mathsf{Z}(X)$}
In the subposet $\mathsf{Z}(X)$, not every subset has a supremum
and an infimum, although every subset (subset of $\mathsf{Z}(X)$) is
bounded by $X$ from above and by $\varnothing$ from below. For instance,
if $X=\bb{N}$ and
\begin{align*}
	\cal{A} & \,=\, \left\lbrace \{2n\} \,:\, n\in \bb{N}\right\rbrace
	\text{ ,}
\end{align*}
%
then any upper bound must contain every even number and must, in particular,
be cofinite (in order to belong to $\mathsf{Z}(\bb{N})$). So there are
(nontrivial) upper bounds, but there is not a smallest one. Similarly, if
\begin{align*}
	\cal{A} & \,=\, \left\lbrace 2\cdot\bb{N} \cup \{n\leq m\} \,:\,
		n\in\bb{N} \right\rbrace
	\text{ ,}
\end{align*}
%
then this set has no infimum. In $\mathsf{Z}(X)$, however, if
$\cal{A}\subset\mathsf{Z}(X)$ is finite, then it admits both a join and a
meet. As it has been mentioned, the poset $\powset{X}$ admits supremums and
infimums for every given subset $\cal{A}\subset\powset{X}$. In particular,
it admits joins and meets for every given pair $A,B\in\powset{X}$, which are
given, respectively, by pairwise union and intersetion of $A$ and $B$. Since
$\mathsf{Z}(X)$ is closed with respect to pairwise unions and intersections,
$\mathsf{Z}(X)$ also admits joins and meets. The poset $\mathsf{Z}(X)$
also admits a least element, $\varnothing$, and a greatest element, $X$.

\subsection{Join-semilattices and Meet-semilattices}
Let $(P.\leq)$ be a partially ordered set. If every pair of elements
$x,y\in P$ has a join $x\boolsum y$ and $P$ has also an element $0\in P$
which is less than every other element of $P$, then the following equalities
hold for every $x,y,z\in P$:
\begin{align}%\begin{equation}
	\label{eq:joinsemilat}
	\begin{split}%\begin{aligned}
	x\boolsum x & \,=\, x \text{ ,}\\
	x\boolsum y & \,=\, y\boolsum x \text{ ,}\\
	x\boolsum (y\boolsum z) & \,=\, (x\boolsum y)\boolsum z \text{ ,}\\
	x\boolsum 0 & \,=\, x
	\text{ .}
	\end{split}%\end{aligned}
\end{align}%\end{equation}
%
Equivalently, we may assume that every finite subset (including the empty
subset) admits a join (supremum): if this is true, then every set containing
exactly two elements admits a join and the fact that there exists an
element $x=\bigboolsum\varnothing$ means that $x$ is less than every other
element of $P$; conversely, if every pair of elements admits a join in $P$,
then, by induction in the number of elements, every finite subset admits a
join, as well, and, since $x\in P$ is less than every other element of $P$,
then, by definition, $x=\bigboolsum\varnothing$. We summarise this paragraph
in the following theorem.

\begin{thmJoinSemilat}\label{thm:joinsemilat}
	Let $(P,\leq)$ be a poset. Such that every finite (or empty) set
	admits a join. Then $(P,\boolsum,0)$ has the structure of a
	commutative monoid in which every element is idempotent, that is to
	say, equations \eqref{eq:joinsemilat} hold.
\end{thmJoinSemilat}

Conversely, triples $(P,\boolsum,0)$ satisfying \eqref{eq:joinsemilat}
can also be characterised in terms of an order relation, as the following
theorem shows.

\begin{thmJoinSemilatChar}\label{thm:joinsemilatchar}
	Let $(P,\boolsum,0)$ be a commutative monoid in which every element
	is an idempotent. Then there exists a unique partial order on $P$
	such that $x\boolsum y$ is the join of $x$ and $y$, and $0$ is the
	least element.
\end{thmJoinSemilatChar}

\begin{proof}
	If $\leq$ is such a partial order on $P$, then $x\leq y$ implies
	$x\boolsum y=\mathsf{join}(\{x,y\})=y$ and, conversely,
	$x\boolsum y=y$ implies $\mathsf{join}(\{x,y\})=y$ and $x\leq y$.

	If we define a binary relation $\leq$ on $P$ by
	$x\leq y$ if $x\boolsum y=y$, then: \textit{(i)} idempotency of
	$\boolsum$ implies reflexivity of $\leq$, \textit{(ii)}
	commutativity implies antisymmetry and \textit{(iii)} associativity
	implies transitivity. So $\leq$ is a partial order on $P$.

	Given $x,y\in P$, associativity and idempotency imply
	$x\boolsum(x\boolsum y)=(x\boolsum x)\boolsum y=x\boolsum y$, so
	$x\leq x\boolsum y$ by definition of $\leq$. Similarly, by
	commutativity, $y\leq x\boolsum y$, so $x\boolsum y$ is an upper bound
	for the subset $\{x,y\}$. If $z\in P$ is such that $x,y\leq z$, then
	$(x\boolsum y)\boolsum z=x\boolsum(y\boolsum z)=z$. This shows that
	$x\boolsum y$ is a least upper bound for $\{x,y\}$, which means that
	$x\boolsum y$ is the join of $x$ and $y$. Finally, since $0$ is the
	neutral element, it is the least element in $P$.
\end{proof}

A set $P$ together with a binary operation $\boolsum$ and a distinguished
element $0$ such that equations \eqref{eq:joinsemilat} hold has the structure
of a \emph{(join-)semilattice}. Theorems \ref{thm:joinsemilat} and
\ref{thm:joinsemilatchar} show that there is a correspondence between posets
$P$ having all finite joins (in particular, we are assuming that
$\bigboolsum\varnothing$ exists, which is equivalent to $P$ having a least
element) and join-semilattices (that is, commutative monoids in which every
element is idempotent):
\begin{center}
\begin{tikzcd}
	\begin{Bmatrix}
		\text{posets }(P,\leq)\text{ having} \\
		\text{all finite joins}
	\end{Bmatrix}
	\arrow[r] &
	\begin{Bmatrix}
		\text{join-semilattices} \\
		(P,\boolsum,0)
	\end{Bmatrix}
\end{tikzcd}
\end{center}
%
The image via this arrow of a poset $(P,\leq)$ is the triple $(P,\boolsum,0)$
consisting of the set $P$, the binary join operation determined by the
order relation $\leq$ and the least element $0=\bigboolsum\varnothing$.
If, conversely, $(P,\boolsum,0)$ is a join-semilattice, then the relation
given by $x\leq y$ if $x\boolsum y=y$ is a partial order on $P$ such that
$x\boolsum y$ is the join of $x$ and $y$ and $0$ is the least element in $P$.
This means that the triple $(P,\boolsum,0)$ is the image of $(P,\leq)$ by
the above arrow. The fact that the order $\leq$ is unique with this property
means that this arrow is a correspondence.

However, this map is not an equivalence in the sense that, although a
join-semilattice homomorphism must be an order-preserving morphism, there
are order-preserving morphisms which do not respect the corresponding
semilattice structure: for example, if $\bb{N}=\{1,\,2,\,\dots\}$
denotes the set of natural numbers with their usual ordering, and if
$\bb{N}_{0}=\bb{N}\cup\{0\}$ with $0\leq n$ for $n\in\bb{N}$, then
the inclusion $\bb{N}\hookrightarrow\bb{N}_{0}$ is order-preserving,
but does not preserve the unit. This implies that the reverse arrow
from join-semilattices to posets with all finite joins, although a
bijection at the level of objects, is merely an injection at the level of
morphisms.

The notions of lower bound, meet and greatest element are dual to those
of upper bound, join and least element. By reversing all the inequalities
implicit in the previous discusion, we arrive at analogous conclusions.

\begin{thmMeetSemilat}\label{thm:meetsemilat}
	Let $(P,\leq)$ be a poset such that every finite subset admits a
	meet. Let $1=\bigboolprod\varnothing$ denote the greatest element
	and let $x\boolprod y$ denote the meet of elements $x$ and $y$ in $P$.
	Then $(P,\boolprod,1)$ has the structure of a commutative monoid
	in which every element is idempotent. This means that the equations
	\begin{equation}
		\label{eq:meetsemilat}
		\begin{aligned}
			x\boolprod x & \,=\, x\text{ ,}\\
			x\boolprod y & \,=\, y\boolprod x\text{ ,}\\
			x\boolprod(y\boolprod z) & \,=\,
				(x\boolprod y\boolprod z)\text{ ,}\\
			x\boolprod 1 & \,=\, x
				\text{ .}
		\end{aligned}
	\end{equation}
	%
	hold for every $x,y,z\in P$.
\end{thmMeetSemilat}

For the sake of symmetry, we shall call a triple $(P,\boolprod,1)$
such that equations \eqref{eq:meetsemilat} hold a \emph{(meet-)semilattice}.

\begin{thmMeetSemilatChar}\label{thm:meetsemilatchar}
	Let $(P,\boolprod,1)$ be a meet-semilattice. Then there exists a
	unique partial order on $P$ such that $x\boolprod y$ is the meet
	of $x$ and $y$, and $1$ is the greates element.
\end{thmMeetSemilatChar}

\subsection{Boolean Algebras}
A \emph{lattice} is a partially ordered set $(P,\leq)$ such that every
pair of elements $x,y\in P$ has a meet and a join $x\boolprod y$ and
$x\boolsum y$, respectively, and there is a least and a greatest element
$0$ and $1$, respectively, in $P$ (or, equivalently, every finite (possibly
empty) subset has a meet and a join in $P$); we denote it by
$(P,\leq,\boolsum,\boolprod,0,1)$.
By theorems \ref{thm:joinsemilatchar} and \ref{thm:meetsemilatchar},
a lattice can be characterised in the following way: let $P$ be a set,
$\boolsum$ and $\boolprod$ binary operations on $P$ and $0$ and $1$
distinguished elements of $P$ such that $(P,\boolsum,0)$ is a
join-semilattice and $(P,\boolprod,1)$ is a meet-semilattice.
Then this defines a lattice structure on $P$, if and only if, in additon,
the induced partial orders are \emph{opposite} to each other, that is to say,
$x\boolsum y=y\Leftrightarrow x\boolprod y=x$, for every $x,y\in P$.
This additional condition is a necessary condition for $P$ to be a poset
when given any of the two partial orders, since in any poset
$\mathsf{join}(x,y)=y\Leftrightarrow\mathsf{meet}(x,y)=x$. Conversely,
if $\leq$ is the partial order on $P$ induced by $(P,\boolsum,0)$, say,
then, on the one hand,
\begin{align*}
	x & \,\leq\, y \quad\Leftrightarrow\quad
	y\,=\,\mathsf{join}(x,y)\,=\,x\boolsum y \quad\Leftrightarrow\quad
	x\,=\,x\boolprod y
	\text{ ,}
\end{align*}
%
which means that, in the order induced by $(P,\boolsum,0)$, $x\boolprod y$
equals $\mathsf{meet}(x,y)$; on the other hand, $x\boolprod 1=x$ for all
$x\in P$ implies $x\boolsum 1=1$ for all $x\in P$ and, thus, $1$ is the
greatest element in $P$. Thus, by uniqueness, the partial order induced by
$(P,\boolsum,0)$ coincides with the (opposite to the) one induced by
$(P,\boolprod,1)$, meaning that $P$ is a poset with join $\boolsum$,
meet $\boolprod$, least element $0$ and greatest element $1$.

\begin{propoAbsorptiveLaws}\label{thm:absorptivelaws}
	Suppose $(P,\boolsum,0)$ and $(P,\boolprod,1)$ are semilattices.
	Then $(P,\boolsum,\boolprod,0,1)$ is a lattice, if and only if
	the \emph{absorptive laws}
	\begin{equation}
		\label{eq:absorptivelaws}
		\begin{aligned}
			x\boolprod (x\boolsum y) & \,=\, x \\
			x\boolsum (x\boolprod y) & \,=\, x
		\end{aligned}
	\end{equation}
	%
	are satisfied for all $x,y\in P$.
\end{propoAbsorptiveLaws}

\begin{proof}
	If the equations \eqref{eq:absorptivelaws} hold, then
	$x\boolsum y=y\Leftrightarrow x\boolprod y=x$. Conversely, in a
	lattice, the absorptive laws hold (take for instance $P=\powset{X}$
	and notice that only the poset-with-finite-joins-and-meets structure
	is needed).
\end{proof}

A \emph{distributive lattice} is a lattice $P$ in which the
\emph{distributive law} holds: for every $x,y,z\in P$,
\begin{equation}
	\label{eq:distlaw}
	\begin{aligned}
		x\boolprod (y\boolsum z) & \,=\,
			(x\boolprod y)\boolsum (x\boolprod z)
	\end{aligned}
\end{equation}

\begin{lemmaDualDistLaw}\label{thm:dualdistlaw}
	If $P$ is a distributive lattice, then the \emph{dual} of the
	distributive law \eqref{eq:distlaw} holds, as well:
	for all $x,y,z\in P$,
	\begin{equation}
		\label{eq:dualdistlaw}
		\begin{aligned}
			x\boolsum (y\boolprod z) & \,=\,
				(x\boolsum y)\boolprod (x\boolsum z)
		\end{aligned}
	\end{equation}
	%
	Conversely, in an arbirary lattice, the identity
	\eqref{eq:dualdistlaw} implies the identity \eqref{eq:distlaw}.
	Thus, a distributive lattice is characterised by either of
	\eqref{eq:distlaw} or \eqref{eq:dualdistlaw}.
\end{lemmaDualDistLaw}

Distributive lattices have the following important property.

\begin{propoDistLat}\label{thm:distlat}
	Let $P$ be a distributive lattice and let $x,y,z\in P$. Then there
	exists at most one element $w\in P$ satisfying both
	\begin{align*}
		w\boolprod x & \,=\, y\quad\text{and} \\
		w\boolsum x & \,=\, z\text{ .}
	\end{align*}
	%
\end{propoDistLat}

Let $x$ be and element of an arbitrary lattice $P$. A \emph{complement} of
$x$ in $P$ is an element $\conj{x}\in P$ satisfying both
$\conj{x}\boolprod x=0$ and $\conj{x}\boolsum x=1$. Complements need not
exist, in general; and, if they do, they need not be unique. Proposition
\ref{thm:distlat} says that in a distributive lattice $P$, if an element $x$
admits a complement, then it must be the only complement in $P$. A
\emph{Boolean algebra} is a distributive lattice $P$ together with an
additional unary operation $\conj{\cdot}:\,P\rightarrow P$ such that,
for every $x\in P$, the element $\conj{x}$ is a complement of $x$.

\begin{remarkBoolAlgIsFullInLat}
	If map between lattices preserves the lattice structure, that is,
	joins, meets, least element and greatest element, then it must
	also preserve the complementation relation. In particular, every
	lattice morphism is a Boolean algebra morphism.
\end{remarkBoolAlgIsFullInLat}

\subsection{Boolean Algebras and Boolean Rings}
Let $P$ be a Boolean algebra. The \emph{symmetric difference} operation
in $P$ is defined as the following binary operation: if $x,y\in P$, let
\begin{align*}
	x + y & \,=\, (x\boolprod\conj{y})\boolsum(\conj{x}\boolprod y)
	\text{ .}
\end{align*}
%
Appealing to the identities \eqref{eq:distlaw} and \eqref{eq:dualdistlaw},
\begin{align*}
	x + y & \,=\, (x\boolsum y)\boolprod(\conj{x}\boolsum\conj{y})
	\text{ ,}
\end{align*}
%
and by existence and uniqueness of complements in a Boolean algebra,
\begin{align*}
	\conj{x}\boolsum\conj{y} & \,=\, \lconj{(x\boolprod y)}
	\text{ .}
\end{align*}
%
So, equivalently,
\begin{align*}
	x + y & \,=\, (x\boolsum y)\boolprod\lconj{(x\boolprod y)}
	\text{ .}
\end{align*}
%

The proof of this equivalence made use of one of the \emph{De Morgan laws}:
\begin{equation}
	\label{eq:demorganlaws}
	\begin{aligned}
		\lconj{(x\boolprod y)} & \,=\, \conj{x}\boolsum\conj{y}
			\text{ ,}\\
		\lconj{(x\boolsum y)}& \,=\, \conj{x}\boolprod\conj{y}
		\text{ .}
	\end{aligned}
\end{equation}
%
Both of these identities hold in a Boolean algebra.

\begin{thmBoolAlgIsBoolRng}\label{thm:boolalgisboolrng}
	Let $P$ be a Boolean algebra, let $\boolsum$, $\boolprod$, $0$ and
	$1$ be its join, meet, least element and greatest element. Let
	$+$ denote its symmetric difference. Then, for every $x,y,z\in P$,
	\begin{align*}
		x\boolprod (y+z) & \,=\, (x\boolprod y) + (x\boolprod z)
			\text{ ,} \\
		x + (y+z) & \,=\, (x+y) + z\text{ ,} \\
		x + y & \,=\, y + x \text{ ,} \\
		x + x & \,=\, 0\quad\text{and} \\
		x + 0 & \,=\, x
		\text{ .}
	\end{align*}
	%
\end{thmBoolAlgIsBoolRng}

These identities, together with
\begin{align*}
	x \boolprod y & \,=\, y\boolprod x \text{ ,} \\
	x\boolprod 1 & \,=\, x\quad\text{and} \\
	x\boolprod x & \,=\, x
	\text{ ,}
\end{align*}
%
show that $(P,+,0)$ is an Abelian group (in which every element
is nilpotent) and that $(P,+,\boolprod,0,1)$ is a commutative
ring with unit in which every element is idempotent.

A \emph{Boolean ring} is a ring with unit $(A,+,\cdot,0,1)$ in which
every element is idempotent, that is, $x^{2}=x\cdot x=x$ for all $x\in A$.
So, \ref{thm:boolalgisboolrng} shows there exists an arrow sending
each Boolean algebra to an associated Boolean ring:
\begin{center}
	\begin{tikzcd}
		\begin{Bmatrix}
			\text{Boolean} \\
			\text{algebras}
		\end{Bmatrix} \arrow[r] &
			\begin{Bmatrix}
				\text{Boolean} \\
				\text{rings}
			\end{Bmatrix} \\
		(P,\boolsum,\boolprod,0,1,\conj{\cdot})\arrow[r, maps to] &
			(P,+,\cdot,0,1)
	\end{tikzcd}
\end{center}
where $+$ is defined as the symmetric difference operation and
$x\cdot y=x\boolprod y$. Since a Boolean algebra morphism preserves
the operations $\boolsum$, $\boolprod$ and $\conj{\cdot}$ it must also
preserve symmetric difference, and since such morphisms preserve greatest
and least elements, they must preserve the neutral elements for addition
and product in the associated Boolean rings. In other words, every morphism
between Boolean algebras \emph{is, too,} a morphism between the corresponding
Boolean rings.

Boolean rings are commutative and every element of a Boolean ring satisfies
$x+x=0$. This implies that $(A,\cdot,1)$ satisfies the identities
\eqref{eq:meetsemilat}. By theorem \ref{thm:meetsemilatchar} there is a
unique partial order on $A$ such that $x\cdot y =\mathsf{meet}(x,y)$ and
$1$ is the greatest element: this order is defined by $x\leq y$ if
$x\cdot y=x$.

\begin{lemmaBoolRngIsDistLat}\label{thm:boolrngisdistlat}
	Let $(A,+,\cdot,0,1)$ be a Boolean ring and let $\leq$ be the
	unique partial order determined by the meet-semilattice
	$(A,\cdot,1)$. Explicitly, $x\leq y$ if $x\cdot y=x$.
	For $x,y\in A$, let $x\boolsum y= x+y+x\cdot y$, and let
	$x\boolprod y=x\cdot y$. Then $0$ is the least element in $A$,
	the following absorptive law holds:
	\begin{align*}
		x\boolprod (x\boolsum y) & \,=\, x
		\text{ ,}
	\end{align*}
	%
	the product $\boolprod$ distributes over $\boolsum$
	and also $x\boolsum y= \mathsf{join}(x,y)$.
\end{lemmaBoolRngIsDistLat}

The previous lemma implies that every Boolean ring has the structure of a
distributive lattice and that this structure is unique, if we want
$x\cdot y=\mathsf{meet}(x,y)$ and $1$ to be the greatest element. Moreover,
since $(1+x)\cdot x=0$ and $(1+x) + x + (1+x)\cdot x =1$, every element
is complemented. This complement must be unique.

\begin{thmBoolRngIsBoolAlg}\label{thm:boolrngisboolalg}
	If $(A,+,\cdot,0,1)$ is a Boolean ring, then, with the partial
	order defined in \ref{thm:boolrngisdistlat} and the unary operation
	$x\mapsto \conj{x}=1+x$, the distributive lattice $A$ is a Boolean
	algebra.
\end{thmBoolRngIsBoolAlg}

Theorem \ref{thm:boolrngisboolalg} shows the existence of an arrow in
the opposite direction to that of the one described after theorem
\ref{thm:boolalgisboolrng}:
\begin{center}
	\begin{tikzcd}
		\begin{Bmatrix}
			\text{Boolean} \\
			\text{rings}
		\end{Bmatrix} \arrow[r] &
			\begin{Bmatrix}
				\text{Boolean} \\
				\text{algebras}
			\end{Bmatrix} \\
		(A,+,\cdot,0,1)\arrow[r, maps to] &
			(A,\boolsum,\boolprod,0,1,\conj{\cdot})
	\end{tikzcd}
\end{center}
where the join is given by $x\boolsum y= x+y+x\cdot y$, the meet is defined
by $x\boolprod y= x\cdot y$, and the complement of and element $x$ is
$\conj{x}=1+x$. Since a Boolean ring morphism by definition preserves sum,
product, zero and one, every such morphism must preserve join, meet, least
element, greatest element and complements in the associated Boolean algebras.
So, every morphism between Boolean rings \emph{is also} a morphism between
the corresponding Boolean algebras. The relation between both arrows is
clarified by the following lemma.

\begin{lemmaSymmDiff}\label{thm:symmdiff}
	If $(A,+,\cdot,0,1)$ is a Boolean ring and $A$ is endowed with
	the Boolean algebra structure discribed in \ref{thm:boolrngisboolalg},
	then the symmetric difference operation is given by:
	\begin{align*}
		(x\boolprod\conj{y})\boolsum(\conj{x}\boolprod y) & \,=\,
			(x\cdot(1+y))\boolsum ((1+x)\cdot y) \\
		& \,=\, x+y
		\text{ .}
	\end{align*}
	%
\end{lemmaSymmDiff}

Thus the two arrows between Boolean algebras and Boolean rings are
inverse of each other, and since every Boolean algebra morphism is a
Boolean ring morphism and \textit{vice versa}, both categories are isomorphic.

\begin{thmBoolAlgIsoBoolRng}\label{thm:boolalgisoboolrng}
	The category of Boolean algebras is isomorphic to the category
	of Boolean rings.
\end{thmBoolAlgIsoBoolRng}

Something that was not mentioned is the fact that $\conj{\conj{x}}=x$
for every element $x$ in a Boolean algebra. This property is related to
the concept of a Heyting algebra.

\paragraph{Example}
Let $\bb{N}$ be the set of natural numbers. On $\powset{\bb{N}}$, define
the following binary relation: $A\sim B$ if
$A\Delta B=(A\setmin B)\cup(B\setmin A)$ is finite. This is an equivalence
relation. Let $[A]$ denote the class of an element $A\in\powset{\bb{N}}$
and let $X$ denote the set of classes. In $X$ define the following order
relation: $[A]\leq[B]$ if $A\setmin B$ is finite (or empty). This definition
does not depend on the choice of representatives $A$ an $B$ for the classes.
This says that a class $[B]$ is greater than another class $[A]$, if $B$
contains every element of $A$, except perhaps a finite number of them
(and that it is strictly greater, if $B$ exceeds $A$ by more than just a
finite number of elements). The set $X$ together with the partial order
$\leq$ is a Boolean algebra.

