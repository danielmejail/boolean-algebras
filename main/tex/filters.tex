\theoremstyle{plain}
\newtheorem{thmSeparation}{Proposition}[section]
\newtheorem{propoLatPrimeIdeal}[thmSeparation]{Proposition}
\newtheorem{thmMaxDisjointIdealIsPrime}[thmSeparation]{Theorem}
\newtheorem{coroDistLatIsConcrete}[thmSeparation]{Corollary}
\newtheorem{propoBoolPrimeIsMax}[thmSeparation]{Proposition}
\newtheorem{propoPrimeIdealIsMax}[thmSeparation]{Proposition}
\newtheorem{lemmaInABooleanAlgebra}[thmSeparation]{Lemma}
\newtheorem{lemmaFIP}[thmSeparation]{Lemma}
\newtheorem{lemmaFilterGeneratedBySubset}[thmSeparation]{Lemma}
\newtheorem{propoPrimeFilterIsMax}[thmSeparation]{Proposition}
\newtheorem{thmMaxDisjointFilterIsPrime}[thmSeparation]{Theorem}
\newtheorem{thmFilterSeparation}[thmSeparation]{Corollary}

\theoremstyle{remark}
\newtheorem*{remarkKerIsIdealIsKer}{Remark}
\newtheorem*{remarkKerIsNotEnough}{Remark}
\newtheorem*{remarkNotAnIdeal}{Remark}
\newtheorem*{remarkPrimeFilterIsPrimeFilter}{Remark}

%------------------

\subsection{Ideals and Filters}
Let $(P,\boolsum,0)$ be a join-semilattice. An \emph{ideal} in $P$ is a
subset $I$ of $P$ such that:
\begin{itemize}
	\item[i] $I$ is a sub-join-semilattice of $P$ and
	\item[ii] for every $x\in I$ and $y\in P$, if $y\leq x$ then
		$y\in I$.
\end{itemize}
%
In contrast to the ring-theoretic notion of ideal, it is not necessary to
include the axiom $0\in I$, for any one of \textit{(i)} and \textit{(ii)}
force $0\in I$. The set of ideals in a join-semilattice is nonempty:
$\{0\}$ and $P$ are ideals.

The first condition in the above definition means that $I$ is a submonoid
of $(P,\boolsum,0)$, that is, if $x,y\in I$ then $x\boolsum y\in I$, too.
This is the analogue of being an additive subgroup. The second condition is
the analogue of being closed under multiplication by elements of the ring.
If $P$ is a Boolean algebra, $y\leq x$ if $y\boolprod x=y$, so the second
condition is translated as being closed under multiplication by elements
of $P$, where multiplication by an element $y\in P$ means
$x\mapsto y\boolprod x$ (this is the actual multiplication in the
corresponding Boolean ring). If now $I\subset P$ satisfies \textit{(i)}
and \textit{(ii)} above, then $I$ is also closed under symmetric difference,
that is, if $x,y\in I$ then $x+y\in I$, where $+$ denotes the symmetric
difference operation in $P$ ($I$ is an ideal in the corresponding Boolean
ring). Conversely, if $I+I\subset I$ and $P\cdot I\subset I$, then
\textit{(i)} and \textit{(ii)} also hold. Thus, in a Boolean algebra, the
ideals according to this definition are exactly the ideals of the
corresponding Boolean ring.

Let $P$ be a join-semilattice with partial order $\leq$. If $x\in P$, the
subset of elements below $x$ is an ideal and is the smallest ideal
containing $x$. We call the ideal
\begin{align*}
	(x):=\{y\in P\,:\,y\leq x\}
\end{align*}
%
the \emph{pricipal ideal} generated by $x$ in $P$. We may also denote this
ideal by $\pideal{x}$.

A \emph{filter} in a meet-semilattice $(P,\boolprod,1)$ is a subset
$F\subset P$ satisfying axioms dual to those defining an ideal:
\begin{itemize}
	\item[i] if $x,y\in F$ then $x\boolprod y\in F$ ($F$ is a
		sub-meet-semilattice) and
	\item[ii] for every $x\in F$ and $y\in P$, if $y\geq x$ then $y\in F$.
\end{itemize}
%
Actually, for a subset $F$ of $P$ to be a sub-meet-semilattice, both
\textit{(i)} and $1\in F$ must hold. But $1\in F$ is implied by \textit{(ii)}.
Thus, every filter contains the greatest element $1$ and is, therefore,
nonempty. The set of filters in a meet-semilattice is nonempty: $\{1\}$ and
$P$ are filters, the trivial filters, in $P$. A \emph{principal filter}
is a filter of the form $\{y\,:\,y\geq x\}$, this filter is said to be
generated by the element $x$. The principal filter generated by $x$ shall
be denoted $(x)$ or $\pfilter{x}$.

\begin{remarkKerIsIdealIsKer}
	Let $P$ and $Q$ be two join-semilattices and let $f:\,P\rightarrow Q$
	be a semilattice morphism. Then the \emph{kernel} of $f$,
	the subset $\{x\in P\,:\,fx=0\}$ is an ideal in $P$. Every ideal
	in a join-semilattice $P$ can be recovered as the kernel of a
	semilattice morphism: given an ideal $I$ in $P$, define a relation
	$\sim_{I}$ on $P$ by $x\sim_{I} y$ if there exist $i,j\in I$ such
	that $x\boolsum i=y\boolsum j$; this is an equivalence relation and
	since $x\sim_{I}y$ implies $x\boolsum z\sim_{I}y\boolsum z$ for
	every $z\in P$, the set of equivalence classes can be made into a
	join-semilattice; the canonical projection from $P$ onto the set
	of equivalence classes is a semilattice morphism with kernel $I$.

	If $P$ is a distributive lattice and $I$ is an ideal in $P$,
	then $I$ is equal to the kernel of some \emph{lattice} morphism:
	we only need to verify that the equivalence relation defined in the
	previous paragraph respects meets, that is
	$x\boolprod z\sim_{I}y\boolprod z$ for every $x,y\in P$ such that
	$x\sim_{I}y$. This follows from distributivity: if $x\boolsum i$
	is equal to $y\boolsum j$, then
	\begin{align*}
		(x\boolsum i)\boolprod z & \,=\, (y\boolsum j)\boolprod z
			\quad\text{and} \\
		(x\boolprod z)\boolsum (i\boolprod z) & \,=\,
			(y\boolprod z)\boolsum (j\boolprod z)
		\text{ .}
	\end{align*}
	%
	The set of classes can be made into a lattice, which, being a
	\emph{quotient} of the distributive lattice $P$, is, then,
	distributive.
\end{remarkKerIsIdealIsKer}

\begin{remarkKerIsNotEnough}
	Surjective lattice or semilattice morphisms are not determined by
	their kernels. Take for example the totally ordered $P=\{0,x,1\}$
	with $x\not =0,1$ and $0\leq x\leq 1$ and the two-element lattice
	$\thetwo=\{0,1\}$. The map $f:\,P\rightarrow\thetwo$ given by
	\begin{align*}
		0\mapsto 0\text{ ,} \quad &
			x\mapsto 1\text{ ,} \quad
			1\mapsto 1
	\end{align*}
	%
	is a lattice homomorphism with kernel $\{0\}$. On the other hand,
	there are no (Boolean) ring homomorphisms from $\{0,x,1\}$ to
	$\{0,1\}$. But this is not the only surjective lattice homomorphism
	from $P$ with trivial kernel: the identity morphism
	$\id:\,P\rightarrow P$ is also such a map. In fact these are the
	only two such morphisms. Although $\ker(\id)=\ker(f)$, the maps
	$f$ and $\id$ can be distinguished by the inverse image of the
	element $1$:
	\begin{align*}
		f^{-1}(1) & \,=\, \left\lbrace y\in P\,:\,f(y)=1\right\rbrace
			\,\not=\,\left\lbrace y\in P\,:\,\id(y)=1\right\rbrace
			\,=\,\id^{-1}(1)
		\text{ .}
	\end{align*}
	%
	In general, if $f:\,P\rightarrow Q$ is a meet-semilattice
	homomorphism, the set $f^{-1}(1)$ is a filter in $P$.
\end{remarkKerIsNotEnough}

As seen from the previous remark, the interaction between ideals and
filters plays an important part in lattice theory.

\begin{propoLatPrimeIdeal}\label{thm:latprimeideal}
	Let $(P,\boolsum,\boolprod,0,1)$ be a lattice and let $I\subset P$
	be an ideal in $P$. The following conditions on $I$ are equivalent:
	\begin{itemize}
		\item[i] the complement of $I$ in $P$ is a filter;
		\item[ii] the greatest element $1$ does not belong to $I$
			and $x\boolprod y\in I$ implies $x\in I$ or $y\in I$;
			and
		\item[iii] $I$ is the kernel of a lattice homomorphism
			$f:\,P\rightarrow\thetwo$.
	\end{itemize}
	%
\end{propoLatPrimeIdeal}

\begin{proof}
	If $F=P\setmin I$ is a filter, then $1\in F$. In particular,
	$1\not\in I$. If also $x\boolprod y\in I$, then
	$x\boolprod y\not\in F$. In particular, it must be true that either
	$x\not\in F$ or that $y\not\in F$, so that $x\in I$ or $y\in I$.

	If $1\not\in I$ and $x\boolprod y\in I$ implies $x\in I$ or $y\in I$,
	then the map $f:\,P\rightarrow\thetwo$ defined by
	\begin{align*}
		f(x) & \,=\,
			\begin{cases}
				0 & \text{ if } x\in I \\
				1 & \text{ if } x\not\in I
			\end{cases}
	\end{align*}
	%
	is a lattice homomorphism. Clearly $f(0)=0$ and $f(1)=1$.
	If $x,y\in I$, then $x\boolsum y\in I$ and
	$f(x\boolsum y)=0=f(x)\boolsum f(y)$. If, on the other hand,
	$x\not\in I$ or $y\not\in I$, the inequality $x,y\leq x\boolsum y$
	implies $x\boolsum y\not\in I$ and
	$f(x)\boolsum f(y)=1=f(x\boolsum y)$. Thus $f$ respects $\boolsum$.
	The additional conditions on $I$ imply that $f$ must also respect
	$\boolprod$.

	Finally, if $I$ is the kernel of a lattice homomorphism
	$f:\,P\rightarrow\thetwo$, then $P\setmin I=f^{-1}(1)$ is a filter.
\end{proof}

\subsection{Prime Ideals and Prime Filters}
Let $(P,\boolsum,\boolprod,0,1)$ be a lattice. An ideal $I\subset P$
satisfying any of the equivalent conditions stated in \ref{thm:latprimeideal}
is said to be a \emph{prime ideal} and its complement $F=P\setmin I$,
which is a filter, a \emph{prime filter}.

Let $I$ be an ideal in the lattice $P$ and let $F$ be a filter in $P$
disjoint from $I$. Applying Zorn's lemma, to the set of ideals which contain
$I$ and are disjoint from $F$, we see that there exists an ideal $M$ in $P$
which is maximal amongst those ideals containing $I$ and disjoint from $F$.

In particular, if $I=0$ and $F$ is any filter, there exists a maximal ideal
amongst those disjoint from $F$.

\begin{thmMaxDisjointIdealIsPrime}\label{thm:maxdisjointidealisprime}
	Let $P$ be distributive lattice and let $F$ be a filter in $P$.
	If $I$ is a maximal ideal amongst those disjoint from $F$, then
	$I$ is prime.
\end{thmMaxDisjointIdealIsPrime}

\begin{proof}
	Since $I\cap F=\varnothing$, $1\not\in I$. Let $x_{1},x_{2}\in P$ be
	such that $x_{1}\boolprod x_{2}\in I$. Let $J_{k}=(I,x_{k})$ be the
	ideal generated by $I$ and $x_{k}$ in $P$. The proof consists in
	checking that $J_{k}$ is the subset of elements of the form
	$i\boolsum(x_{k}\boolprod y)$ with $i\in I$ and $y\in P$
	(which is an ideal); and then showing that at least one of $J_{1}$
	and $J_{2}$ must be disjoint from $F$. Thus $J_{k}\supset I$ for some
	$k$. Maximality of $I$ implies $J_{k}=I$ and $x_{k}\in I$.
\end{proof}

Applying theorem \ref{thm:maxdisjointidealisprime} with $F=\{1\}$, so
$I$ is a \emph{maximal proper ideal}, we may conclude that every maximal
proper ideal is prime.

\begin{remarkNotAnIdeal}
	Although $\{1\}$ is a filter and $\{0\}$ is an ideal, the subset
	$P\setmin\{1\}$ is not an ideal and $P\setmin\{0\}$ is not a filter,
	unless $\{1\}$ is a prime filter (tautologically) and $\{0\}$ is a
	prime ideal, respectively.
\end{remarkNotAnIdeal}

\begin{thmSeparation}\label{thm:separation}
	Let $P$ be a distributive lattice and let $x,y\in P$ with
	$x\not\geq y$. Then there exists a lattice homomorphism
	$f:\,P\rightarrow\thetwo$ with $f(x)=0$ and $f(y)=1$.
\end{thmSeparation}

\begin{proof}
	Let $I=\pideal{x}$ and let $F=\pfilter{y}$. The condition
	$y\not\geq x$ implies $I\cap F=\varnothing$. By Zorn's lemma,
	there exists a maximal ideal $I'$ amongst those containing $I$
	and disjoint from $F$. By \ref{thm:maxdisjointidealisprime},
	the ideal $I'$ is a prime ideal and by \ref{thm:latprimeideal},
	$I'$ is the kernel of some lattice homomorphism
	$f:\,P\rightarrow\thetwo$. Since $x\in I'$ and $y\not\in I'$,
	$f(x)=0$ and $f(y)=1$.
\end{proof}

\begin{coroDistLatIsConcrete}\label{thm:distlatisconcrete}
	Any distributive lattice $P$ is isomorphic to a sublattice
	of the power set $\powset{X}$ of some set $X$.
\end{coroDistLatIsConcrete}

\begin{proof}
	Take $X$ to be the set of homomorphisms $P\rightarrow\thetwo$.
\end{proof}

\begin{propoBoolPrimeIsMax}\label{thm:boolprimeismax}
	Let $I$ be an ideal in a Boolean algebra $P$. The following
	conditions are equivalent:
	\begin{itemize}
		\item[i] $I$ is prime;
		\item[ii] for every $x\in P$, either $x\in I$ xor
			$\conj{x}\in I$; and
		\item[iii] $I$ is (proper) maximal.
	\end{itemize}
	%
\end{propoBoolPrimeIsMax}

\begin{proof}
	Suppose first that $I$ is prime. Since $x\boolprod\conj{x}=0$
	and $0\in I$, at least one of $x$ and $\conj{x}$ must belong to
	$I$, but since $x\boolsum\conj{x}=1$ and $1\not\in I$, at least
	one must lie outside $I$.

	Assume now that for every $x\in P$ either $x\in I$ or $\conj{x}\in I$
	but not both. Since $0\in I$ and $1=\conj{0}$, we have $1\not\in I$
	and $I$ is proper. If $J$ is some other ideal strictly containing $I$,
	then there exists $x\in J\setmin I$. Since $x\not\in I$, it must
	be the case that $\conj{x}\in I$ and $1=x\boolsum\conj{x}\in J$
	and $J=P$. Thus, the only ideal containing $I$ strictly is $P$
	and $I$ is a maximal proper ideal.

	Since a Boolean algebra is, by definition, a distributive lattice,
	every maximal ideal in $P$ must be prime by
	\ref{thm:maxdisjointidealisprime}.
\end{proof}
In fact, \textit{i} implies \textit{ii} is true in an arbitrary
lattice, as long as $x$ admits a complement, and \textit{ii} implies
\textit{iii} is	true in any complemented lattice.

\begin{propoPrimeIdealIsMax}\label{thm:primeidealismax}
	Let $I$ be a (proper) ideal in a lattice $P$. Assume $I$ is a prime
	ideal. If $x\in P$ and $z$ is a complement for $x$ in $P$ (possibly
	not the only one), then $x\in I$ or $z\in I$, but not both.

	If $P$ is a complemented lattice and $I$ is a (proper) ideal such
	that, for every $x\in P$, either $x\in I$ or $\conj{x}\in I$, then
	$I$ is maximal.
\end{propoPrimeIdealIsMax}

\subsection{Ultrafilters}
The above discussion was centered around the notion of ideal. We have 
defined prime ideals as those whose complement was a filter and maximal
(proper) ideals as those which are not strictly contained in another proper
ideal. Applying Zorn's lemma we have arrived to the conclusion that maximal
ideals exist. Furthermore, given an ideal $I$ and a filter $F$ disjoint from
$I$, there exists a maximal ideal amongst those containing $I$ and disjoint
from $F$. We have also proved that, in a distributive lattice, every maximal
ideal is prime and that, in a Boolean algebra, the notions of prime ideal
and of maximal ideal agree and such ideals are characterised by the following
condition: let $I\subset P$ be an ideal in a Boolean algebra $P$, then $I$
is a maximal (equivalently, prime) ideal, if and only if for every $x\in P$,
either $x\in I$ or $\conj{x}\in I$, but not both. The only ideal in a Boolean
algebra containing both an element $x$ and its complement $\conj{x}$ is the
trivial ideal $P$. In this subsection we develop the analogous notion to that
of maximal ideal on the side of filters.

Let $(P,\boolsum,\boolprod,0,1)$ be a lattice. Let $F\subset P$ be a filter
and let $I$ be an ideal in $P$ disjoint from $F$. Applying Zorn's lemma to
the set of filters containing $F$ and disjoint from $I$, we may conclude
that there exists a maximal filter amongst those extending $F$ and remaining
disjoint from $I$. We shall say that a filter $F'$ \emph{extends} a filter
$F$, if $F\subset F'$. If we take $F=\{1\}$, we arrive at the notion
of a maximal filter disjoint from an ideal $I$. In particular, if $I=\{0\}$,
we get maximal (proper) filters. An \emph{ultrafilter} in a lattice $P$
is a filter $F$ in $P$ which is a maximal proper filter.

\begin{remarkPrimeFilterIsPrimeFilter}
	Let $P$ be a lattice. We have seen in \ref{thm:latprimeideal} that,
	if $I\subset P$ is an ideal, then the following two conditions are
	equivalent:
	\begin{itemize}
		\item[i] the complement of $I$ in $P$ is a filter;
		\item[ii] $1\not\in I$ and $x\boolprod y\in I$ implies
			$x\in I$ or $y\in I$.
	\end{itemize}
	%
	(These can be shown to be equivalent independently of the
	third condition in \ref{thm:latprimeideal}).
	Such were the prime ideals. A prime filter was defined as the
	complement of a prime ideal. We may, however, state the analagous
	proposition proposition for filters. Let $F\subset P$ be a filter
	and consider the following conditions on $F$.
	\begin{itemize}
		\item[i] The complement of $F$ in $P$ is an ideal;
		\item[ii] $0\not\in F$ and $x\boolsum y\in F$ implies
			$x\in F$ or $y\in F$.
	\end{itemize}
	%
	These two conditions on the filter $F$ are equivalent. Thus, the
	notion of a prime filter as a filter which satisfies any of the
	above equivalen conditions agrees with the notion of a prime 
	filter as the complement of a prime ideal.

	In a Boolean algebra, the prime ideals are precisely the maximal
	ideals. So, the notions of prime filter and maximal filter are quite
	different. Perhaps there is an analogous construction to that of
	Boolean algebras in which thet notion of prime and maximal filter
	agree.

	Anyhow, the matter of ultrafilters in Boolean algebras is not
	directly settled by the results in the previous subsections.
\end{remarkPrimeFilterIsPrimeFilter}

We shall restrict our attention to Boolean algebras, although some definitions
make sense for arbitrary lattices.

\begin{lemmaInABooleanAlgebra}\label{thm:inabooleanalgebra}
	In a Boolean algebra, $x\boolprod\conj{y}=0$, if and only if
	$x\leq y$. Dually, in a Boolean algebra, $x\boolsum\conj{y}=1$,
	if and only if $x\geq y$.
\end{lemmaInABooleanAlgebra}

A subset $A$ of a lattice $P$ is said to have the \emph{finite intersection %
property}, if the meet of any finite subset of $A$ is not equal to $0$. In
symbols, $A$ has the finite intersection property, if, for every
$a_{1},\,\dots,\,a_{m}\in A$,
\begin{align*}
	a_{1}\boolprod\,\cdots\,\boolprod a_{m} &
		\,=\,\bigboolprod\left\lbrace a_{1},\,\dots,\,a_{m}
			\right\rbrace
		\,\not=\, 0
\end{align*}
%

\begin{lemmaFIP}\label{thm:fip}
	Let $P$ be a Boolean algebra.
	\begin{itemize}
		\item[(a)] If $A\subset P$ has the finite intesection
			property, then, for any element $x\in P$, either
			$A\cup\{x\}$ or $A\cup\{\conj{x}\}$ has the finite
			intersection property.
		\item[(b)] Let $\{A_{i}\}_{i}$ be a chain of subsets
			of $P$ totally ordered by inclusion. If every $A_{i}$
			has the finite intersection property, then their
			union $\bigcup_{i}\,A_{i}$ has the finite intersection
			property, as well.
	\end{itemize}
	%
\end{lemmaFIP}

Let $P$ be a lattice and let $A\subset P$ be a subset. Define
$\langle A\rangle$ to be the set of elements of $P$ greater than some
element of $A$ and $|A|$ to be the set consisting of the meets in $P$
of the finite subsets of $A$:
\begin{align*}
	\langle A\rangle & \,:=\,\left\lbrace y\in P\,:\, x\leq y
				\text{ for some } x\in A\right\rbrace \\
		& \,=\,\bigcup_{x\in A}\,\pfilter{x} \text{ .} \\
	|A| & \,:=\,\left\lbrace a_{1}\boolprod\,\cdots\,\boolprod a_{m}\,:\,
			a_{1},\,\dots,\,a_{m}\in A\right\rbrace \\
		& \,=\,\left\lbrace \bigboolprod A_{1}\,:\,
				A_{1}\subset A,\,\# A_{1}<\infty\right\rbrace
	\text{ .}
\end{align*}
%
A \emph{base} for a filter $F$ in $P$ is a subset $A$ such that
$\langle A\rangle = F$. A \emph{subbase} for $F$ is a set $A$ such that
$|A|$ is a base for $F$. We shall say that $A$ \emph{generates} $F$,
if $F=\langle |A|\rangle$.

\begin{lemmaFilterGeneratedBySubset}\label{thm:filtergeneratedbysubset}
	Let $P$ be a Boolean algebra and let $A\subset P$ be a subset.
	Then
	\begin{itemize}
		\item[(a)] The set $\langle |A|\rangle$ is a filter in $P$,
			though not necessarily a proper one;
		\item[(b)] the filter $\langle |A|\rangle$ is the smallest
			filter containing $A$; and
		\item[(c)] $\langle |A|\rangle$ is a proper filter, if and
			only if $A$ has thet finite intersection property.
	\end{itemize}
	%
\end{lemmaFilterGeneratedBySubset}

We call $\langle |A|\rangle$ the \emph{filter generated} by $A$. By
item \textit{(c)} of \ref{thm:filtergeneratedbysubset}, a subset $A$ in a
Boolean algebra can be extended to a proper filter --and, in particular, to
an ultrafilter--, if and only if $A$ has the finite intersection property.

\begin{proof}
	An element $x\in P$ belongs to $\langle |A|\rangle$, if and only
	if there exists a finite subset
	$A_{1}=\{a_{1},\,\dots,\,a_{m}\}\subset A$ such that
	\begin{align*}
		\bigboolprod A_{1} & \,=\,
			a_{1}\boolprod\,\cdots\,\boolprod a_{m} \,\leq\, x
		\text{ .}
	\end{align*}
	%
	Given a subset $B\subset P$, the set $|B|$ consisting of all
	finite meets of elements of $B$ (meets of finite subsets of $B$),
	is $\boolprod$-closed, since, by associativity,
	\begin{align*}
		\Big(\bigboolprod B_{1}\Big)\boolprod
			\Big(\bigboolprod B_{2}\Big) &
			\,=\,\bigboolprod (B_{1}\cup B_{2})
		\text{ .}
	\end{align*}
	%
	Thus, $\langle |A|\rangle$ satisfies the axioms for filters.

	Let $F$ be a filter containing $A$. Then, since $F$ is
	$\boolprod$-closed, the set $|A|$ of meets of finite subests of $A$
	is contained in $F$. But if $x\in F$, the filter $\pfilter{x}$ is
	contained in $F$. Therefore
	\begin{align*}
		F & \,\supset\,\bigcup_{
			\begin{smallmatrix}
				A_{1}\subset A\\
				\# A_{1}<\infty
			\end{smallmatrix}}\,\pfilter{\bigboolprod A_{1}}
			\,=\,\langle |A|\rangle
		\text{ .}
	\end{align*}
	%

	Finally, if $A$ does not have the finite intersection property,
	there exists a finite subset $A_{1}\subset A$ such that
	$\bigboolprod A_{1}=0$. In particular, $0$ lies in
	$|A|\subset\langle |A|\rangle$ and, being a filter,
	$\langle |A|\rangle=P$. If, conversely, $\langle |A|\rangle$ is not a
	proper filter, then $0\in\langle |A|\rangle$, which, by definition,
	means there exists a finite subset $A_{1}\subset A$ such that
	$\bigboolprod A_{1}\leq 0$ and so $A$ does not have the finite
	intersection property.
\end{proof}

The following proposition is dual to \ref{thm:primeidealismax}.

\begin{propoPrimeFilterIsMax}\label{thm:primefilterismax}
	Let $F$ be a (proper) filter in a lattice $P$. Assume $F$ is a prime
	filter. If $x\in P$ and $z$ is a complement for $x$ in $P$ (possibly
	not the only one), then $x\in F$ or $z\in F$, but not both.

	If $P$ is a complemented lattice and $F$ is a (proper) filter such
	that, for every $x\in P$, either $x\in F$ or $\conj{x}\in F$, then
	$F$ is maximal.
\end{propoPrimeFilterIsMax}

The following theorem is dual to \ref{thm:maxdisjointidealisprime}.

\begin{thmMaxDisjointFilterIsPrime}\label{thm:maxdisjointfilterisprime}
	Let $P$ be a distributive lattice and let $I$ be an ideal in $P$.
	If $F$ is a maximal filter amongst those disjoint from $I$, then
	$F$ is prime.
\end{thmMaxDisjointFilterIsPrime}

\begin{proof}
	The proof is dual to the proof of \ref{thm:maxdisjointidealisprime}.
	Since $I\cap F=\varnothing$, $0\not\in F$. Let $x_{1},x_{2}\in P$
	be such that $x_{1}\boolsum x_{2}\in F$. Let
	\begin{align*}
		G_{1} & \,=\,\langle |F,x_{1}|\rangle
		\text{ .}
	\end{align*}
	%
	be the filter generated by $F$ and $x_{1}$ and let
	\begin{align*}
		E_{1} & \,=\, \left\lbrace i\boolprod(x_{1}\boolsum y)\,:\,
			i\in F,\,y\in P\right\rbrace
		\text{ .}
	\end{align*}
	%
	If $i,j\in F$ and $y,z\in P$,
	\begin{align*}
		[i\boolprod (x_{1}\boolsum y)]\boolprod
			[j\boolprod (x_{1}\boolsum z)] & \,=\,
			(i\boolprod j)\boolprod [(x_{1}\boolsum y)\boolprod
						(x_{1}\boolsum z)]
			& \,=\,(i\boolprod j)\boolprod
				(x_{1}\boolsum (y\boolprod z))
		\text{ .}
	\end{align*}
	%
	Since $i\boolprod j\in F$, we conclude that $E_{1}$ is
	$\boolprod$-closed. If $z\geq i\boolprod (x_{1}\boolsum y)$, then
	\begin{align*}
		z & \,=\, z\boolsum (i\boolprod (x_{1}\boolsum y))
			\,=\, (z\boolsum i)\boolprod
				(z\boolsum (x_{1}\boolsum y))
		\text{ .}
	\end{align*}
	%
	Since $F$ is a filter and $i\in F$, $z\boolsum i\in F$ and
	$z\in E_{1}$. Thus we see that $E_{1}$ is a filter containing
	$F$ and $x_{1}$. Since $G_{1}$ is the smallest such filter,
	$G_{1}\subset E_{1}$. On the other hand, $E_{1}\subset G_{1}$, for
	if $i\in F$ and $z:=i\boolprod (x_{1}\boolsum y)\in E_{1}$, then
	$z\geq i\boolprod x_{1}$ and $\pfilter{i\boolprod x_{1}}\subset G_{1}$.
	Similarly, $G_{2}=E_{2}$.

	We now prove that at least one of $G_{1}$ and $G_{2}$ is disjoint
	from $I$. If not, there exist $i,j\in F$ and $y,z\in P$ such that
	both $i\boolprod(x_{1}\boolsum y)$ and $j\boolprod (x_{2}\boolsum z)$
	belong to $I$. Since $I$ is an ideal, the element
	\begin{align*}
		(i\boolprod (x_{1}\boolsum y))\boolsum
			(j\boolprod (x_{2}\boolsum z)) & \,=\,
		(i\boolsum j)\boolprod(i\boolsum x_{2}\boolsum z)\boolprod
			(j\boolsum x_{1}\boolsum y)\boolprod
			(x_{1}\boolsum x_{2}\boolsum y\boolsum z)
	\end{align*}
	%
	belongs to $I$. But it also belongs to $F$, for the term on the
	right is a finite meet of elements of $F$. This contradicts
	$I\cap F=\varnothing$. Therefore, at least one of $G_{1}$ and $G_{2}$
	must be disjoint from $I$ and hence $G_{k}\supset F$ for some $k$.
	But $F$ is maximal, so $G_{k}=F$ and $x_{k}\in F$ for some $k$.
\end{proof}

In particular, every filter in a distributive lattice can be extended to an
ultrafilter. Since in a Boolean algebra the only subsets which can be
extended to proper filters are the subsets with the finite intersection
property, we conclude that every subset of a Boolean algebra with the
finite intersection property can be extended to an ultrafilter. In particular,
if $x\in P$ is an element in a Boolean algebra, there is an ultrafilter
containing $x$. More precisely, the following corollary is true.

\begin{thmFilterSeparation}\label{thm:filterseparation}
	If $x,y$ are distinct elements of a Boolean algebra $P$, then
	there exists an ultrafilter containing one but not the other.
\end{thmFilterSeparation}

\begin{proof}
	Since $x\not = y$, either $x\not\leq y$ or $x\not\geq y$.
	Assume, without loss of generality, that $x\not\geq y$ holds.
	Then, by \ref{thm:inabooleanalgebra}, $x\boolsum\conj{y}\not =1$
	or, equivalently, $y\boolprod\conj{x}\not =0$. Then the set
	$\{y,\conj{x}\}$ has the finite intersection property and can
	be extended to an ultrafilter $F$ in $P$. Thus, $y\in F$ and
	$\conj{x}\in F$. Since $F$ is, in particular, a proper filter,
	it must be te case that $x\not\in F$.
\end{proof}

This corollary is, in a sense, similar to \ref{thm:separation}.
In a distributive lattice $P$ we can separate distinct elements by lattice
homomorphisms into the two-element lattice $\thetwo=\{0,1\}$. Given a lattice
homomorphism $f:\,P\rightarrow\thetwo$ there is a descomposition of $P$
as a disjoint union of an ideal and a filter:
\begin{align*}
	P & \,=\,f^{-1}(0)\cup f^{-1}(1)
	\text{ .}
\end{align*}
%
Although homomorphisms are typically related to ideals through their kernels,
in this case we may as well treat this result as relating to filters.
\ref{thm:filterseparation}, however, is more precise in its conclusion:
although $f^{-1}(1)$ is a maximal filter amongst those disjoint from
$\pideal{x}$ and containing $\pfilter{y}$, it may not be an ultrafilter.
\ref{thm:filterseparation} concludes that two distinct elements in a Boolean
algebra can be \emph{maximally separated}, that is, separated by an
ultrafilter.

Finally, we state the following consequence of \ref{thm:primefilterismax}
and \ref{thm:maxdisjointfilterisprime}, which is dual to
\ref{thm:boolprimeismax}. It says that the notions of prime filter and of
maximal filter (ultrafilter) agree in a Boolean algebra.

